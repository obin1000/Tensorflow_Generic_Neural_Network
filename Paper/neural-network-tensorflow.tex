%%%%%%%%%%%%%%%%%%%%%%%%%%%%%%%%%%%%%%%%%
% Stylish Article
% LaTeX Template
% Version 2.0 (13/4/14)
%
% This template has been downloaded from:
% http://www.LaTeXTemplates.com
%
% Original author:
% Mathias Legrand (legrand.mathias@gmail.com)
%
% License:
% CC BY-NC-SA 3.0 (http://creativecommons.org/licenses/by-nc-sa/3.0/)
%
%%%%%%%%%%%%%%%%%%%%%%%%%%%%%%%%%%%%%%%%%

%----------------------------------------------------------------------------------------
%	PACKAGES AND OTHER DOCUMENT CONFIGURATIONS
%----------------------------------------------------------------------------------------

\documentclass[fleqn,10pt]{SelfArx} % Document font size and equations flushed left

\usepackage{lipsum} % Required to insert dummy text. To be removed otherwise
\usepackage{listings}
\usepackage{graphicx}
\graphicspath{Paper/images/}

%----------------------------------------------------------------------------------------
%	COLUMNS
%----------------------------------------------------------------------------------------

\setlength{\columnsep}{0.55cm} % Distance between the two columns of text
\setlength{\fboxrule}{0.75pt} % Width of the border around the abstract

%----------------------------------------------------------------------------------------
%	COLORS
%----------------------------------------------------------------------------------------

\definecolor{color1}{RGB}{0,0,90} % Color of the article title and sections
\definecolor{color2}{RGB}{0,20,20} % Color of the boxes behind the abstract and headings

%----------------------------------------------------------------------------------------
%	HYPERLINKS
%----------------------------------------------------------------------------------------

\usepackage{hyperref} % Required for hyperlinks
\hypersetup{hidelinks,colorlinks,breaklinks=true,urlcolor=color2,citecolor=color1,linkcolor=color1,bookmarksopen=false,pdftitle={Title},pdfauthor={Author}}

%----------------------------------------------------------------------------------------
%	ARTICLE INFORMATION
%----------------------------------------------------------------------------------------

\JournalInfo{Artificial neural networks, January, 2020} % Journal information
\Archive{Go neural or go home} % Additional notes (e.g. copyright, DOI, review/research article)

\PaperTitle{Neural network recognizes clothing} % Article title

\Authors{Robin Vonk, Michel Rummens} % Authors
% \textsuperscript{1}*
% \affiliation{\textsuperscript{1}\textit{Department of Biology, University of Examples, London, United Kingdom}} % Author affiliation
% \affiliation{\textsuperscript{2}\textit{Department of Chemistry, University of Examples, London, United Kingdom}} % Author affiliation
% \affiliation{*\textbf{Corresponding author}: john@smith.com} % Corresponding author

\Keywords{} % Keywords - if you don't want any simply remove all the text between the curly brackets
\newcommand{\keywordname}{Keywords} % Defines the keywords heading name

%----------------------------------------------------------------------------------------
%	ABSTRACT
%----------------------------------------------------------------------------------------

\Abstract{}

%----------------------------------------------------------------------------------------

\begin{document}

\flushbottom % Makes all text pages the same height

\maketitle

\tableofcontents % Print the contents section

\thispagestyle{empty} % Removes page numbering from the first page

%----------------------------------------------------------------------------------------
%	ARTICLE CONTENTS
%----------------------------------------------------------------------------------------

\section*{Introduction} % The \section*{} command stops section numbering

\addcontentsline{toc}{section}{Introduction} % Adds this section to the table of contents

%------------------------------------------------


% \begin{figure}[ht]\centering
% \includegraphics[width=\linewidth]{results}
% \caption{In-text Picture}
% \label{fig:results}
% \end{figure}

% Reference to Figure \ref{fig:results}.

%------------------------------------------------
\section{What is Tensorflow?}
TensorFlow is an end-to-end open source platform for machine learning. It has a comprehensive, flexible ecosystem of tools, libraries and community resources that lets researchers push the state-of-the-art in ML and developers easily build and deploy ML powered applications.\\ \\
TensorFlow offers multiple levels of abstraction so you can choose the right one for your needs. Build and train models by using the high-level Keras API, which makes getting started with TensorFlow and machine learning easy.\\ \\
If you need more flexibility, eager execution allows for immediate iteration and intuitive debugging. For large ML training tasks, use the Distribution Strategy API for distributed training on different hardware configurations without changing the model definition. \cite{Tensorflow} \\ \\
So whether you are simply trying out neural networks - as we were - or planning on using it in production, Tensorflow has got you covered!

\section{What we built in Tensorflow}
For the first step, we build an neural network that is trained with the MNIST data-set to recognize pieces of clothing. 
We used a web-scraper \cite{WebScraping} to retrieve 

\section{The data set generator}

\subsection{Google images}



\section{The neural network}

\subsection{input layer}

\subsection{hidden layer}

\subsection{output layer}


%------------------------------------------------
% \phantomsection
% \section*{Acknowledgments} % The \section*{} command stops section numbering

% \addcontentsline{toc}{section}{Acknowledgments} % Adds this section to the table of contents

% So long and thanks for all the fish \cite{Figueredo:2009dg}.

%----------------------------------------------------------------------------------------
%	REFERENCE LIST
%----------------------------------------------------------------------------------------
\phantomsection
\bibliographystyle{unsrt}
\bibliography{References}

%----------------------------------------------------------------------------------------

\end{document}